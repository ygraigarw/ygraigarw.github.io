%\documentclass[11pt,a4paper,sans]{moderncv}
\documentclass[11pt,a4paper]{moderncv}
\moderncvstyle{classic}
\moderncvcolor{green}
\usepackage[scale=0.75]{geometry}
\usepackage{tgpagella}
\renewcommand{\sfdefault}{\rmdefault}

%%%%%%%%%%%%%%%%%%%%%%%%%%%%%%%%%%%%%%%%%%%%%%%%%%%%%%%%%%%
%Page size
\oddsidemargin  -0.7in
\evensidemargin -0.7in
\textwidth      7.6in
\headheight     0.0in
\topmargin      -1.0in
\textheight     10.5in
%%%%%%%%%%%%%%%%%%%%%%%%%%%%%%%%%%%%%%%%%%%%%%%%%%%%%%%%%%%

% personal data
\name{Philip}{Jonathan}
\address{+44 7899 821 260}
\email{p.jonathan@lancaster.ac.uk}                   % optional, remove / comment the line if not wanted
\homepage{www.lancaster.ac.uk/$\sim$jonathan}       % optional, remove / comment the line if not wanted
\extrainfo{(also philip.jonathan@shell.com)\\ November 2023}                   % optional, remove / comment the line if not wanted
\photo[64pt][0.4pt]{PhilipJonathan2018.png}       % optional, remove / comment the line if not wanted; '64pt' is the height the picture must be resized to, 0.4pt is the thickness of the frame around it (put it to 0pt for no frame) and 'picture' is the name of the picture file
%\quote{Some quote}                                 % optional, remove / comment the line if not wanted

\begin{document}
\makecvtitle

\vspace{-30pt}

\section{In brief}

\setlength{\parskip}{0.5em}

Philip Jonathan is a statistical modeller, Chief Statistician at Shell and Chair of Environmental Statistics and Data Science in the Department of Mathematics and Statistics at Lancaster University, UK (both on a part-time basis from April 2023). His current research is reported at www.lancs.ac.uk/$\sim$jonathan. He has more than 30 years experience in statistical modelling of physical systems, particularly the ocean environment, and a background in applied mathematics and physics. He is particularly interested in characterising extreme ocean environments, and the development of methodology for good design and re-analysis of offshore and coastal infrastructure. He has published on the development of methods for large-scale non-stationary marginal extreme value modelling with multidimensional covariates, and non-stationary extensions of multivariate, spatial and time-series extreme value methods, and applied these widely. He has developed and refined estimators summarising the tails of distributions, including return values, associated values and environmental design contours in the presence of uncertainty. His long-standing interests include Bayesian uncertainty analysis for complex systems, Bayesian inference for optimal design and probabilistic inversion in remote sensing, multivariate prediction and computational statistics generally. He has also published on ion physics, chemical kinetics, multivariate analysis, chemometrics and quantitative structure--activity relationships, single pixel cameras, wave radar measurement, time-series and change-point analysis, and ancient writing systems. Illustrative references include:

\setlength{\parskip}{0.2em}

K. Ewans and \textbf{P. Jonathan} Uncertainties in estimating the effect of climate change on 100-year return value for significant wave height. Ocean Eng. 272:113840. arXiv:2212.11049, 2023.

J. P. Grainger, A. Sykulski, K. Ewans, H. F. Hansen and \textbf{P. Jonathan}. A multivariate pseudo-likelihood approach to estimating ocean wave direction models. J. R. Stat. Soc. C 118: 1373-1384, 2023.

S.Tendijck, E. Eastoe, J. Tawn, D. Randell and \textbf{P. Jonathan}. Modelling the extremes of bivariate mixture distributions with application to oceanographic data. J. Amer. Statist. Assoc. 118:1373--1384, 2023.

C. Roberts, O. Shorttle, K. Mandel, M. Jones, R. Ijzermans, B. Hirst and \textbf{P. Jonathan}. Enhanced monitoring of atmospheric methane from space with hierarchical Bayesian inference. Environ. Res. Lett. 17:064037, 2022. arXiv:2111.12486.

R. Shooter, J. A. Tawn, E. Ross  and \textbf{P. Jonathan}. Basin-wide conditional spatial extremes for severe ocean storms. Extremes 24:241--265, 2021.

E. Ross, M. Kereszturi, M. van Nee, D. Randell and \textbf{P. Jonathan}. On the spatial dependence of extreme ocean storm sea states. Ocean Eng. 145:1--14, 2017.

D. Randell, K. Turnbull, K. Ewans and \textbf{P. Jonathan}. Bayesian inference for non-stationary marginal extremes. Environmetrics 27:439--450, 2016.

M. J. Jones, M. Goldstein, \textbf{P. Jonathan}, and D. Randell. Bayes linear analysis for Bayesian optimal experimental design. Journal of Statistical Planning and Inference, 171:115--129, 2015.

\textbf{P. Jonathan}, K. C. Ewans, and D. Randell. Non-stationary conditional extremes of northern North Sea storm characteristics. Environmetrics, 125:172--188, 2014.

R. Lee, \textbf{P. Jonathan}, and P. R. Ziman. Pictish symbols revealed as a written language through application of Shannon entropy. Proc. R. Soc. A, 467:309--313, 2011.

\textbf{P. Jonathan}, W. J. Krzanowski, and W. V. McCarthy. On the use of cross-validation to assess performance in multivariate prediction. Statistics and Computing, 10:209--229, 2000.

P. H. Taylor, \textbf{P. Jonathan}, and L. A. Harland. Time-domain simulation of jack-up dynamics using the extremes of a random process. Journal of Vibrations and Accoustics, 119:624, 1995.

M. Stone and \textbf{P. Jonathan}. Statistical thinking and technique for QSAR and related studies. part I : General theory. J. Chemometr. 7:455, 1993.

E. Y. Kamber, \textbf{P. Jonathan}, A. G. Brenton, and J. H. Beynon. Single-electron capture by $Ar^{++}$ from atomic and molecular targets. J. Phys. B: Atom. Mol. Physics, 20:4129, 1987.

\setlength{\parskip}{0.5em}

\section{Employment}
\cventry{2018--date}{Chair, Environmental Statistics and Data Science}{Lancaster University}{Lancaster, UK}{}{}
\cventry{2018--date}{Chief Statistician}{Shell Projects \& Technology}{London, UK}{}{}
\cventry{2004--2018}{Head, Statistics \& Data Science}{Shell Projects \& Technology}{London, UK}{}{}
\cventry{1996--2004}{Statistician}{Shell Thornton}{Chester, UK}{}{}
\cventry{1993--1996}{Mathematician}{Shell Research}{Rijswijk, NL}{}{}
\cventry{1992--1993}{Consulting Statistician}{Shell Development Company}{Houston, USA}{}{}
\cventry{1988--1992}{Mathematician}{Shell Research}{Sittingbourne, UK}{}{}
\cventry{1987--1988}{Process investigator}{Morganite Electrical Carbon}{Swansea, UK}{}{}

\section{Languages}
\cvitem{Welsh}{Mother tongue.}
\cvitem{English}{Fluent.}
\cvitem{Czech}{Conversational.}

\section{Academic qualifications, honours, accreditations and affiliations}
\cventry{2021}{Greenfield Industrial Medallist}{Royal Statistical Society, UK}{}{}{}
\cventry{2008--2018}{Honorary Research Fellow}{Mathematics \& Statistics}{Lancaster University, UK}{}{}
\cventry{2000--2003}{Visiting Fellow}{Chemical Engineering}{Newcastle University, UK}{}{}
\cventry{1999}{Chartered Statistician}{Royal Statistical Society}{UK}{}{}
\cventry{1984--1987}{Ph.D.}{Ion Physics}{Swansea University, UK}{}{}
\cventry{1981--1984}{B.Sc.}{Applied Mathematics}{Swansea University, UK}{(Charles Prize in Mathematics, 1983; Olroyd Prize in Applied Mathematics, 1984)}{}

\section{Research interests}
\cvitem{Extremes}{Extreme value analysis for meteorological-oceanographic applications.}
\cvitem{Uncertainty}{Uncertainty quantification in models for complex systems.}
\cvitem{Inversion}{Probabilistic inversion in remote sensing.}

\section{Academic supervision}
\cventry{2022--date}{T. Newman}{Ph.D., Department of Mathematics \& Statistics, Lancaster University, UK}{}{}{}
\cventry{2022--date}{M. Speers}{Ph.D., Department of Mathematics \& Statistics, Lancaster University, UK}{}{}{}
\cventry{2021--2022}{D. Dodd}{Department of Mathematics \& Statistics, Lancaster University, UK}{}{}{}
\cventry{2019--2022}{J. Grainger}{Ph.D., Department of Mathematics \& Statistics, Lancaster University, UK}{}{}{}
\cventry{2019--2022}{S. Tendijck}{Ph.D., Department of Mathematics \& Statistics, Lancaster University, UK}{}{}{}
\cventry{2016--2020}{R. Shooter}{Ph.D., Department of Mathematics \& Statistics, Lancaster University, UK}{}{}{}
\cventry{2014--2018}{E. Zanini}{Ph.D., Department of Mathematics \& Statistics, Lancaster University, UK}{}{}{}
\cventry{2013--2017}{M. Kereszturi}{Ph.D., Department of Mathematics \& Statistics, Lancaster University, UK}{}{}{}
\cventry{2012--2017}{M. J. Jones}{Ph.D., School of Mathematical Sciences, Durham University, UK}{}{}{}
\cventry{2012--2015}{B. Pickering}{Ph.D., Department of Mathematics \& Statistics, Lancaster University, UK}{}{}{}
\cventry{2011--2012}{D. Reeve}{M.Res., Department of Mathematics \& Statistics, Lancaster University, UK}{}{}{}
\cventry{2010--2015}{R. Towe}{Ph.D., Department of Mathematics \& Statistics, Lancaster University, UK}{}{}{}
\cventry{2010--2023}{Y. Wu}{Ph.D., Department of Mathematics \& Statistics, Lancaster University, UK}{}{}{}
\cventry{2009--2012}{J. Wadsworth}{Ph.D., Department of Mathematics \& Statistics, Lancaster University, UK}{}{}{}
\cventry{2008--2011}{R. Killick}{Ph.D., Department of Mathematics \& Statistics, Lancaster University, UK}{}{}{}
\cventry{2006--2010}{D. Randell}{Ph.D., School of Mathematical Sciences, Durham University, UK}{}{}{}
\cventry{2003--2006}{G. Hardman}{Ph.D., School of Mathematical Sciences, Durham University, UK}{}{}{}
\cventry{2000--2003}{J. Little}{Ph.D., School of Mathematical Sciences, Durham University, UK}{}{}{}
\cventry{1999--2002}{J. Badcock}{Ph.D., School of Mathematical Sciences, Exeter University, UK}{}{}{}
\cventry{1997--2000}{E. Kaskavelis}{Ph.D., Centre for Process Analytics and Control Technology, U. Newcastle, UK}{}{}{}

\section{Other academic activities}

\cvitem{Recent talks and seminars}{Oceanographic extremes (Metoffice, Exeter, 2022; Wales Mathematics Colloquium, Gregynnog, 2022; Royal Statistical Society Conference, 2021; ISSC-ITTC, 2021; Waves SIG, 2020; OMAE Conference, 2020; Lancaster extremes workshop, 2019; Marine Centre Wales, Bangor, 2018; Extreme Value Analysis \& Natural Hazards, Southampton, 2017); Remote sensing (Wales Mathematics Colloquium, Gregynnog, 2022; Data-centric engineering, 2021); Emerging themes in data science (Emerging Applications Section, RSS, London, 2017); Spatial extremes (Extreme Value Analysis \& Natural Hazards, Southampton, 2017); Real-time data and applications (StatScale workshop, Lancaster, 2017).} 

\cvitem{Academic leadership}{Member of REF 2021 Mathematics Sub-panel (2018-21); Advisory Board for Mathematical Sciences (Durham, 2016-2017); StatScale External Advisory Board (Mathematics and Statistics, Lancaster, 2016-2018); STOR-i External Advisory Board (Mathematics and Statistics, Lancaster, 2009-2018).}

\cvitem{External examination of PhDs}{Mathematical Sciences, (Durham, 2020); Department of Civil Engineering and Environmental Science (Imperial College, 2017); School of Chemical Engineering and Advanced Materials, Newcastle, 2016); Department of Mathematical Sciences (Norwegian University of Science and Technology, Trondheim, 2015); Institute for Transport Studies (Leeds, 2011); Centre for Process Analytics and Control Technology (Newcastle, 2003). Also acted as Internal Examiner for Lancaster PhDs since 2018 on numerous occasions.}

%\section{Publications}
%
%Publications listed at \texttt{www.mendeley.com/profiles/philip-jonathan3/publications}. \\
%Recent papers at \texttt{www.lancaster.ac.uk/$\sim$jonathan}.
%
%\end{document}

%\nocite{*}
%\bibliographystyle{plain}
%\bibliography{PhilipJonathan}                        % 'publications' is the name of a BibTeX file

\section{Publications}
Publications are sorted first by year (in reverse), then by author and finally if necessary by title. Citation H-index (Google Scholar, accessed December 2022) is 33 (and 24 since 2017). Total number of citations is 3739 (1910 since 2017). 
\setlength{\parskip}{0.2em}
\vspace{5pt}

2024: E. B. L. Mackay, C. J. R. Murphy-Barltrop and \textbf{P. Jonathan}. The SPAR model: a new paradigm for multivariate extremes. Application to joint distributions of metocean variables. (Submitted to OMAE24 in January 2024, preprint at www.lancs.ac.uk/$\sim$jonathan).

2024: C. J. R. Murphy-Barltrop, E. Mackay and \textbf{P. Jonathan}. Inference for multivariate extremes via a semi-parametric angular-radial model. (Submitted to Extremes in January 2024, preprint at www.lancs.ac.uk/$\sim$jonathan).  arXiv:2401.07259.

2024: R. Paleja, O. Ekhorutomwen, M. Jones, John Ayoola, R. Pitchumani and \textbf{P. Jonathan}. Reducing the CO$_2$ footprint at an LNG asset with replicate trains using operational data-driven analysis. A case study on end flash vessels. (Accepted by Data-Centric Engineering in June 2024, preprint at www.lancs.ac.uk/$\sim$jonathan).

2024: K. Sando, R. Wada, J. Rohmer and \textbf{P. Jonathan}. Multivariate spatial and spatio-temporal models for extreme tropical cyclone seas. (Accepted by Ocean Engineering in June 2024, preprint at www.lancs.ac.uk/$\sim$jonathan.)

2024: M. Speers, D. Randell, J. A. Tawn and \textbf{P. Jonathan}. Estimating metocean environments associated with extreme structural response. (Submitted to Ocean Engineering in April 2024, preprint at www.lancs.ac.uk/$\sim$jonathan.)

2024: S. Tendijck, D. Randell, G. Feld and \textbf{P. Jonathan}. Uncertainties in return values from non-stationary extreme value analysis of peaks over threshold using the generalised Pareto distribution. (Submitted to Ocean Engineering in June 2024, preprint at www.lancs.ac.uk/$\sim$jonathan.)

2024: R. Towe, E. Ross, D. Randell and \textbf{P. Jonathan}. covXtreme : MATLAB software for non-stationary penalised piecewise constant marginal and conditional extreme value models. Environ. Model. Softw. 177:106035. arXiv:2309:17295.

2024: E. J. Watts, R. Rees, T. M. Gernon, \textbf{P. Jonathan}, D. Keir, R. N. Taylor, M. Siegburg, E. L. Chambers, C. Pagli, M. J. Cooper, A. Michalik, J. A. Milton and T. K Hincks. Afar triple junction fed by single, asymmetric mantle upwelling. (Submitted to Nature Geoscience in April 2024).

2023:  A.M. Barlow, E. Mackay, E. Eastoe and \textbf{P. Jonathan}. A penalised piecewise-linear model for non-stationary extreme value analysis of peaks over threshold. Ocean Eng. 267:113265. arXiv:2201.03915.

2023: K. Ewans and \textbf{P. Jonathan} Is the 100-year return value for significant wave height increasing in the Tasman Sea? (Proc. 42th Int. Conf. on Ocean, Offshore \& Arctic Engineering, Mebourne; preprint at  www.lancs.ac.uk/$\sim$jonathan.)

2023: K. Ewans and \textbf{P. Jonathan} Uncertainties in estimating the effect of climate change on 100-year return value for significant wave height. Ocean Eng. 272:113840. arXiv:2212.11049.

2023: J. P. Grainger, A. Sykulski, K. Ewans, H. F. Hansen and \textbf{P. Jonathan}. A multivariate pseudo-likelihood approach to estimating ocean wave direction models. J. R. Stat. Soc. C. 72:544-565.  arXiv:2202.03773.

2023: E. Mackay and \textbf{P. Jonathan}. Modelling multivariate extremes through angular-radial decomposition of the density function. (Submitted to arXiv in December 2023, submitted to Extremes in February 2024) arXiv:2310.12711.

2023: C. Roberts, R. Ijzermans, D. Randell, M.Jones, \textbf{P. Jonathan}, K. Mandel, B. Hirst and O. Shorttle. Avoiding methane emission rate underestimates when using the divergence method. Environ. Res. Lett. 18:114033. arXiv:2304.10303.

2023: S.Tendijck, E. Eastoe, J. Tawn, D. Randell and \textbf{P. Jonathan}. Modelling the extremes of bivariate mixture distributions with application to oceanographic data. J. Amer. Statist. Assoc. 118:1373--1384.

2023: S. Tendijck, \textbf{P. Jonathan}, D. Randell and J. A. Tawn. Temporal evolution of the extreme excursions of multivariate kth order Markov processes with application to oceanographic data. (Accepted by Environmetrics, November 2023, preprint at www.lancs.ac.uk/$\sim$jonathan). arXiv:2302.14501.

2023: S. Tendijck, J. A. Tawn and \textbf{P. Jonathan}. Extremal characteristics of conditional models. Extremes 26:139–156. arXiv:2202.11673.

2023: R. Towe, D. Randell, J. Kensler, G. Feld and \textbf{P. Jonathan}. Estimation of associated values from conditional extreme value models. Ocean Eng. 272:113808.

2022: A. Raby, G. Bullock, \textbf{P. Jonathan}, D. Randell and C. Whittaker. On wave impact pressure variability. Coastal Eng. 177:104168. 

2022: C. Roberts, O. Shorttle, K. Mandel, M. Jones, R. Ijzermans, B. Hirst and \textbf{P. Jonathan}. Enhanced monitoring of atmospheric methane from space with hierarchical Bayesian inference. Environ. Res. Lett. 17:064037. arXiv:2111.12486.

2022: K. Sando, R. Wada, J. Rohmer, S. Lecacheux and \textbf{P. Jonathan}. Estimating joint extremes of significant wave height and wind speed for tropical cyclones. OMAE2022-79888 Proc. 41st Int. Conf. on Ocean, Offshore \& Arctic Engineering. (Preprint at www.lancs.ac.uk/$\sim$jonathan.)

2022: R. Shooter, E. Ross, A. Ribal, I.R. Young and \textbf{P. Jonathan}. Multivariate spatial conditional extremes for extreme ocean environments. Ocean Eng. 247:110647. arXiv:2201.10451.

2022: R. Wada, J. Rohmer, Y. Krien and \textbf{P. Jonathan}. Statistical estimation of spatial wave extremes for tropical cyclones from small data samples: validation of the STM-E approach using long-term synthetic cyclone data for the Caribbean Sea. Nat. Hazards Earth Sys., 22, 431-444. 

2021: P. A. Carling, \textbf{P. Jonathan} and S. Teng. Fitting Limit Lines (envelope curves) to spreads of geoenvironmental data. Prog. Phys. Geog.: Earth \& Env. 46:272-290.

2021: J. P. Grainger, A. Sykulski, \textbf{P. Jonathan} and K. Ewans. Estimating the parameters of ocean wave spectra. Ocean Eng. 229:108934. arXiv:2008.10437.

2021: A. F. Haselsteiner, R. G. Coe, L. Manuel, W. Chai, B. Leira, G. Clarindo, C. Guedes Soares, A. Hannesdottir, N. Dimitrov, A. Sander, J.-H. Ohlendorf, K.-D. Thoben, G. de Hauteclocque, E. Mackay, \textbf{P. Jonathan}, C. Qiao, A. Myers, A. Rode, A. Hildebrandt, B. Schmidt, E. Vanem, A. B. Huseby. A benchmarking exercise for environmental contours. Ocean Eng. 236:109504.

2021: \textbf{P. Jonathan}. Cynrychioliad amharamedrig ar gyfer cydnewidynnau aml-ddimensiynol mewn modelau gwerthoedd eithafol (A non-parametric representation for multi-dimensional covariates in an extreme value model, with oceanographic application), Gwerddon 33:68-84. 

2021: \textbf{P. Jonathan}, D. Randell, J. Wadsworth and J. Tawn. On return values for extreme seas. Ocean Eng. 220:107725.

2021: M. J. Jones, M. Goldstein, D. Randell and \textbf{P. Jonathan}. Bayes linear analysis for ordinary differential equations. Comput. Statist. Data Anal. 161:107228.

2021: E. Konzen, C. Neves and \textbf{P. Jonathan}. Modelling non-stationary extremes of storm severity: a tale of two approaches. Environmetrics 32:e2667. arXiv:2005.13490.

2021: E. Mackay,  \textbf{P. Jonathan}. Sampling properties and empirical estimates of extreme events. Ocean Eng. 239:109791.

2021: E. Mackay, G. de Hauteclocque, E. Vanem and \textbf{P. Jonathan}. The effect of serial correlation in environmental conditions on estimates of extreme events. Ocean Eng. 242:11092.

2021: R. Shooter, E. Ross, A. Ribal, I.R. Young and \textbf{P. Jonathan}. Spatial dependence of extreme seas in the North East Atlantic from satellite altimeter measurements. Environmetrics 32:e2674.

2021: R. Shooter, J. A. Tawn, E. Ross  and \textbf{P. Jonathan}. Basin-wide conditional spatial extremes for severe ocean storms. Extremes 24:241--265.

2021: R. Towe, E. Zanini, D. Randell, G. Feld and \textbf{P. Jonathan}. Efficient estimation of distributional properties of extreme seas from a hierarchical description applied to calculation of un-manning and other weather-related operational windows. Ocean Eng. 238:109642.

2020: K. Ewans, M. Christou, S. Ilic and \textbf{P. Jonathan}. Identifying higher-order interactions in wave time-series. J. Offshore Mech. Arctic Eng. 143:021201-1.

2020: K. Ewans and \textbf{P. Jonathan}. Extreme conditions. In I.R. Young and A. Babanin, editors, Ocean wave dynamics. World Scientific, pp 271-319.

2020: H. F. Hansen, D. Randell, A. R. Zeeberg and \textbf{P. Jonathan}. Directional-seasonal extreme value analysis of North Sea storm conditions. Ocean Eng. 195:106665. 

2020: E. Mackay and \textbf{P. Jonathan}. Assessment of return value estimates from stationary and non-stationary extreme value models. Ocean Eng. 207:107406.

2020: E. Mackay and \textbf{P. Jonathan}. Estimation of environmental coutours using a block resampling method. Proc. 39th Int. Conf. on Ocean, Offshore \& Arctic Engineering.

2020: E. Ross, O. C. Astrup, E. Bitner-Gregersen, N. Bunn, G. Feld, B. Gouldby, A. Huseby, Y. Liu,  D. Randell, E. Vanem and \textbf{P. Jonathan}. On environmental contours for marine and coastal design. Ocean Eng. 195:106194. arXiv:1812.07886.

2020: M. Schubert, Y. Wu,  J. Tychsen, M. H. Faber, J. D. Sorensen and \textbf{P. Jonathan}. On the distribution of maximum wave and crest height at intermediate water depths. Ocean Eng. 217:107485.

2020: R. Wada, \textbf{P. Jonathan}, T. Waseda. Spatial features of extreme waves in the Gulf of Mexico. Proc. 39th Int. Conf. on Ocean, Offshore \& Arctic Engineering.

2020: E. Zanini, E. Eastoe, M. J. Jones, D. Randell and \textbf{P. Jonathan}. Flexible covariate representations for extremes. Environmetrics 31:e2624.

2019: A. Anokhin, D. Randell, E. Ross and \textbf{P. Jonathan}. Spatial and seasonal variability of metocean design criteria in the Southern South China Sea from covariate extreme value analysis, Proc. 38th Int. Conf. on Ocean, Offshore \& Arctic Engineering.

2019: G. Feld, D. Randell and \textbf{P. Jonathan}. On the estimation and application of directional design criteria, Proc. 38th Int. Conf. on Ocean, Offshore \& Arctic Engineering. 

2019: G. Feld, D. Randell, E. Ross and \textbf{P. Jonathan}. Design conditions for waves and water levels using extreme value analysis with covariates. Ocean Eng. 173:851-866.

2019: R. Shooter, E. Ross, J. A. Tawn and \textbf{P. Jonathan}. On spatial conditional extremes for ocean storm severity. Environmetrics 2019:e2562.

2019: E. Vanem, B. Guo, E. Ross and \textbf{P. Jonathan}. Comparing different contour methods with response-based methods for extreme ship response analysis. Mar. Struct. 69:102680.

2019: R. Wada, \textbf{P. Jonathan}, T. Waseda and S. Fan. Estimating extreme waves in the Gulf of Mexico using a simple spatial extremes model, Proc. 38th Int. Conf. on Ocean, Offshore \& Arctic Engineering.

2018: M. J. Jones, M. Goldstein, \textbf{P. Jonathan}, and D. Randell. Bayes linear analysis of sequential optimal design problems. Electron. J. Stat. 12:4002--4031.

2018: M. J. Jones, H. F. Hansen, A. R. Zeeberg, D. Randell, and \textbf{P. Jonathan}. Uncertainty quantification in estimation of ocean environmental return values. Coastal Eng. 141:36--51.

2018: E. Ross, S. Sam, G. Feld, D. Randell and \textbf{P. Jonathan}. Estimating surge in extreme North Sea storms. Ocean Eng. 154:430--444.

2018: S. Tendijck, E. Ross, D. Randell, and \textbf{P. Jonathan}. A model for the directional evolution of severe ocean storms. Environmetrics 30:e2541.

2018: R. Wada, T. Waseda and \textbf{P. Jonathan}. A simple spatial model for extreme tropical cyclone seas.  Ocean Eng. 169:315-325.

2017: A. Brown, W. Gorter, L. Vanderschuren, P. Tromans, \textbf{P. Jonathan}, P. Verlaan. OMAE2017-61005: Design approach for turret moored vessels in highly variable squall conditions. ASME. International Conference on Offshore Mechanics and Arctic Engineering, Volume 3A: Structures, Safety and Reliability, V03AT02A049.

2017: B. Hirst, D. Randell, M. Jones, \textbf{P. Jonathan}, B. King and M. Dean. A new technique for monitoring the atmosphere above onshore carbon storage projects that can estimate the locations and mass emission rates of detected sources.  Energy Procedia 114: 3716--3728.

2017: P. Northrop, N. Attalides and \textbf{P. Jonathan}. Cross-validatory extreme value threshold selection and uncertainty with application to wave heights. J. Roy. Statist. Soc. C, 66:93--120.

2017: E. Ross, M. Kereszturi, M. van Nee, D. Randell and \textbf{P. Jonathan}. On the spatial dependence of extreme ocean storm sea states. Ocean Eng. 145:1--14.

2017: E. Ross, D. Randell, K. Ewans, G. Feld and \textbf{P. Jonathan}. Efficient estimation of return value distributions from non-stationary marginal extreme value models using Bayesian inference. Ocean Eng. 142:315--328.

2016: M. J. Jones, D. Randell, K. Ewans, and \textbf{P. Jonathan}. Statistics of extreme ocean environments: non-stationary inference for directionality and other covariate effects. Ocean Eng. 119:30--46.

2016: M. Kereszturi, J. Tawn, and \textbf{P. Jonathan}. Assessing extremal dependence of North Sea storm severity. Ocean Eng. 118:242--259.

2016: P. Northrop, \textbf{P. Jonathan}, and D. Randell. Threshold modeling of nonstationary extremes. In D. Dey and J. Yan, editors, Extreme Value Modeling and Risk Analysis: Methods and Applications, pages 87--108. Chapman and Hall / CRC.

2016: R. Raghupathi, D. Randell, and \textbf{P. Jonathan}. Consistent design criteria for South China Sea with a large-scale extreme value model. Offshore Technology Conference Asia, Kuala Lumpur, OTC-26668.

2016: L. Raghupathi, D. Randell, K. Ewans, and \textbf{P. Jonathan}. Fast computation of large scale marginal extremes with multi-dimensional covariates. Comput. Statist. Data Anal. 95:243--258.

2016: L. Raghupathi, D. Randell, K. Ewans, and \textbf{P. Jonathan}. OMAE2016-54355: Non-stationary estimation of joint design criteria with a multivariate conditional extremes approach. Proc. 35nd Conf. Offshore Mech. Arct. Eng.

2016: L. Raghupathi, D. Randell, E. Ross, K. Ewans, and \textbf{P. Jonathan}. Multi-dimensional predictive analytics for risk estimation of extreme events. (Big Data Foundations Workshop, IEEE High-Performance Computing, Data and Analytics Conference, HIPC2016).

2016: D. Randell, K. Turnbull, K. Ewans, and \textbf{P. Jonathan}. Bayesian inference for non-stationary marginal extremes. Environmetrics, 27:439--450.

2016: M. Vogel, J. Hanson, S. Fan, G. Z. Forristall, Y. Li, R. Fratantonio, and \textbf{P. Jonathan}. Efficient environmental and structural response analysis by clustering of directional wave spectra. Offshore Technology Conference, Houston, OTC-27039.

2016: R. Wada, T. Waseda, and \textbf{P. Jonathan}. Extreme value estimation using the likelihood-weighted method. Ocean Eng. 124:241--251.

2016: Y. Wu, D. Randell, M. Christou, Kevin Ewans, and P Jonathan. On the distribution of wave heights in shallow water. Coastal Eng. 111:39--49.

2015: M. J. Jones, M. Goldstein, \textbf{P. Jonathan}, and D. Randell. Bayes linear analysis for Bayesian optimal experimental design. J. Stat. Plan. Inference, 171:115--129.

2015: D. Randell, G. Feld, K. Ewans, and \textbf{P. Jonathan}. Distributions of return values for ocean wave characteristics in the South China Sea using directional-seasonal extreme value analysis. Environmetrics, 26:442--450.

2015: R. P. N. Rao, R. Lee, N. Yadav, M. Vahia, \textbf{P. Jonathan}, and P. Ziman. On statistical measures and ancient writing systems. Language 91:198--205.

2014: K. Ewans and \textbf{P. Jonathan}. Evaluating environmental joint extremes for the offshore industry. J. Marine Syst. 130:124--130.

2014: K. Ewans and \textbf{P. Jonathan}. OTC-25036: Recent advances in the analysis of extreme metocean events. Proc. Offshore Technology Conference, Kuala Lumpur, Malaysia (also at www.lancs.ac.uk/$\sim$jonathan). 

2014: K. Ewans, G. Feld, and \textbf{P. Jonathan}. On wave radar measurement. Ocean Dynamics, 64:1281--1303. 

2014: G. Feld, D. Randell, Y. Wu, K. Ewans, and \textbf{P. Jonathan}. OMAE2014-23157: Estimation of storm peak and intra-storm directional-seasonal design conditions in the North Sea. J. Offshore Mech. Arct. Eng. 137:021102.

2014: \textbf{P. Jonathan}, K. Ewans, and J. Flynn. On the estimation of ocean engineering design contours. ASME J. Offshore Mech. Arct. Eng. 136:041101.

2014: \textbf{P. Jonathan}, K. C. Ewans, and D. Randell. Non-stationary conditional extremes of northern North Sea storm characteristics. Environmetrics, 125:172--188.

2014: \textbf{P. Jonathan}, D. Randell, Y. Wu, and K. Ewans. Return level estimation from non-stationary spatial data exhibiting multidimensional covariate effects. Ocean Eng. 88:520--532.

2014: J. Muyau, K. Ewans, and \textbf{P. Jonathan}. OTC-24904: Short-term variability of wind measurements in South China Sea. Proc. Offshore Technology Conference, Kuala Lumpur, Malaysia (also at www.lancs.ac.uk/$\sim$jonathan).

2014: D. Randell, M Goldstein, and \textbf{P. Jonathan}. Bayes linear variance structure learning for inspection of large scale physical systems. Proc. Inst. Mech. Eng. O, 228:3--18.

2014: D. Randell, E. Zanini, M. Vogel, K. Ewans, and \textbf{P. Jonathan}. OMAE2014-23156: Omnidirectional return values for storm severity from directional extreme value models: the effect of physical environment and sample size. Proc. 33nd Conf. Offshore Mech. Arct. Eng.

2014: H. Yu, J. Dauwels, and \textbf{P. Jonathan}. Extreme value graphical models with multivariate copulas. IEEE Transactions on Signal Processing, 62:5734--5747.

2013: E. Eastoe, S. Koukoulas, and \textbf{P. Jonathan}. Statistical measures of extremal dependence illustrated using measured sea surface elevations from a neighbourhood of coastal locations. Ocean Eng. 62:68--77.

2013: B. Hirst, \textbf{P. Jonathan}, F. Gonzalez del Cueto, D. Randell, and O. Kosut. Locating and quantifying gas emission sources using remotely obtained concentration data. Atmospheric Environ. 74:141--158.

2013: R. Killick, I. A. Eckley, and \textbf{P. Jonathan}. A wavelet-based approach for detecting changes in second order structure within nonstationary time series. Electronic Journal of Statistics, 7:1167--1183.

2013: \textbf{P. Jonathan} and K. Ewans. Statistical modelling of extreme ocean environments with implications for marine design : a review. Ocean Eng. 62:91--109.

2013: \textbf{P. Jonathan}, K. C. Ewans, and D. Randell. Joint modelling of environmental parameters for extreme sea states incorporating covariate effects. Coastal Eng. 79:22--31.

2013: D. Randell, Y. Wu, \textbf{P. Jonathan}, and K. Ewans. OMAE2013-10187: Modelling covariate effects in extremes of storm severity on the Australian North West Shelf. Proc. 32nd Conf. Offshore Mech. Arct. Eng.

2013: R. Towe, E. Eastoe, J. Tawn, Y. Wu, and \textbf{P. Jonathan}. OMAE2013-10154: The extremal dependence of storm severity, wind speed and surface level pressure in the northern North Sea. Proc. 32nd Conf. Offshore Mech. Arct. Eng.

2013: S. S. Welsh, M. P. Edgar, R. Bowman, \textbf{P. Jonathan}, B. Sun, M. J. Padgett. Fast full-color computational imaging with single-pixel detectors. Optics Express 21:23068-23074.

2013: S. S. Welsh, M. P. Edgar, \textbf{P. Jonathan}, B. Sun, and M. J. Padgett. Multi-wavelength compressive computational ghost imaging. Proc. SPIE 8618, Emerging Digital Micromirror Device Based Systems and Applications V.

2012: \textbf{P. Jonathan}, K. C. Ewans, and J. Flynn. Joint modelling of vertical profiles of large ocean currents. Ocean Eng. 42:195--204.

2012: D.T. Reeve, D. Randell, K.C.Ewans, and \textbf{P. Jonathan}. Accommodating measurement scale uncertainty in extreme value analysis of North Sea storm severity. Ocean Eng. 53:164--176.

2012: H. Yu, Z. Choo, J. Dauwels, \textbf{P. Jonathan}, and Q. Zhou. Modeling spatially-dependent extreme events with Markov random field priors. In Proc. ISIT 2012 - IEEE Symposium on Information Technology, Boston, MA.

2012: H. Yu, Z. Choo, W. I. T. Uy, J. Dauwels, and \textbf{P. Jonathan}. Modeling extreme events in spatial domain by copula graphical models. In Proc. 15th International Conference on Information Fusion, Singapore.

2011: \textbf{P. Jonathan} and K. Ewans. A spatiodirectional model for extreme waves in the Gulf of Mexico. ASME J. Offshore Mech. Arct. Eng. 133:011601.

2011: \textbf{P. Jonathan} and K. Ewans. Discussion of On the use of discrete seasonal and directional models for the estimation of extreme wave conditions by Edward B.L. Mackay, Peter G. Challenor, Abubakr S. Bahaj [Ocean Eng. 37:425-442]. Ocean Eng. 38:205.

2011: \textbf{P. Jonathan} and K. Ewans. Modelling the seasonality of extreme waves in the Gulf of Mexico. J. Offshore Mech. Arct. Eng. 133:021104.

2011: R. Lee, \textbf{P. Jonathan}, and P. R. Ziman. Pictish symbols revealed as a written language through application of Shannon entropy. Proc. R. Soc. A, 467:309--313.

2011: P. Northrop and \textbf{P. Jonathan}. Threshold modelling of spatially-dependent non-stationary extremes with application to hurricane-induced wave heights. Environmetrics, 22:799--809.

2010: \textbf{P. Jonathan}, J. Flynn, and K. C. Ewans. Joint modelling of wave spectral parameters for extreme sea states. Ocean Eng. 37:1070--1080.

2010: R. Killick, I. Eckley, K. Ewans, and \textbf{P. Jonathan}. Detection of changes in the characteristics of oceanographic time-series using changepoint analysis. Ocean Eng. 37:1120--1126.

2010: D. Randell, M. Goldstein, G. Hardman, and \textbf{P. Jonathan}. Bayesian linear inspection planning for large scale physical systems. Proc. IMechE Part O: Proc. Inst. Mech. Eng. O, 224:333--345.

2010: J. L. Wadsworth, J. A. Tawn, and \textbf{P. Jonathan}. Accounting for choice of measurement scale in extreme value modelling. Ann. App. Stat. 4:1558--1578.

2008: K. C. Ewans and \textbf{P. Jonathan}. The effect of directionality on Northern North Sea extreme wave design criteria. J. Offshore Mech. Arct. Eng. 130:10.

2008: \textbf{P. Jonathan}, K. C. Ewans, and G. Z. Forristall. Statistical estimation of extreme ocean environments: The requirement for modelling directionality and other covariate effects. Ocean Eng. 35:1211--1225.

2007: \textbf{P. Jonathan} and K. C. Ewans. The effect of directionality on extreme wave design criteria. Ocean Eng. 34:1977--1994.

2007: \textbf{P. Jonathan} and K. C. Ewans. Uncertainties in extreme wave height estimates for hurricane dominated regions. J. Offshore Mech. Arct. Eng. 129:300--305.

2007: L. C. Thomson, B. Hirst, G. Gibson, S. Gilespie, \textbf{P. Jonathan}, K. D. Skeldon, and M. J. Padgett. An improved algorithm for locating a gas source using inverse methods. Atmospheric Environ. 41:1128--1134.

2004: J. Badcock, T. Bailey, W. J. Krzanowski, and \textbf{P. Jonathan}. Two projection methods for use in the analysis of multivariate process data with an illustration in petrochemical production. Technometrics, 46:392--403.

2004: J. Little, M. Goldstein, and \textbf{P. Jonathan}. Efficient Bayesian sampling inspection for industrial processes based on transformed spatio-temporal data. Statistical Modelling, 4:299.

2004: J. Little, M. Goldstein, \textbf{P. Jonathan}, and K. den Heijer. Spatio-temporal modelling of corrosion in an industrial furnace. Applied Stochastic Models in Business and Industry, 20:219--238.

2003: A. G. Brenton and \textbf{P. Jonathan}. Calculation of the effect of collisional broadening on high-resolution translational energy loss spectra. Int. J. Mass Spec. Ion Proc. 230:185--192.

2001: E. Kaskavelis, \textbf{P. Jonathan}, E. B. Martin, and A. J. Morris. Statistical analysis of catalyst degradation in a semi-continuous chemical production process. J. Chemometr. 15:665--683.

2000: \textbf{P. Jonathan}, W. J. Krzanowski, and W. V. McCarthy. On the use of cross-validation to assess performance in multivariate prediction. Statistics and Computing, 10:209--229.

1999: P. R. Fisk and \textbf{P. Jonathan}. A new approach to prediction of diffusion coefficients. In: Comprehensive Chemical Kinetics 37: Applications to kinetic modelling. 

1998: C. Elsinghorst, P. Groeneboom, \textbf{P. Jonathan}, L. Smulders, and P.H. Taylor. Extreme value analysis of North Sea storm severity. J. Offshore Mech. Arct. Eng. 120:177--183.

1997: \textbf{P. Jonathan} and P. H. Taylor. On irregular, non-linear waves in a spread sea. J. Offshore Mech. Arct. Eng. 119:37--41.

1996: L. A. Harland, J. H. Vugts, \textbf{P. Jonathan}, and P. H. Taylor. Extreme responses of non-linear dynamic systems using constrained simulations. Proc. 15th Conf. Offshore Mech. Arct. Eng.

1996: \textbf{P. Jonathan}, W. V. McCarthy, and A. M. I. Roberts. Discriminant analysis with singular covariance matrices. A method incorporating cross-validation and efficient randomized permutation tests. J. Chemometr. 10:189-213.

1996: \textbf{P. Jonathan} and P. H. Taylor. Wave-induced loads on fixed offshore structures. an assessment of load variability and bias. Proc. 15th Conf. Offshore Mech. Arct. Eng.

1995: \textbf{P. Jonathan} and P. H. Taylor. Ar frig y don (on the crest of a wave). Delta (Journal of the Welsh Scientific Society).

1995: W. J. Krzanowski, \textbf{P. Jonathan}, W. V. McCarthy, and M. R. Thomas. Discriminant analysis with singular covariance matrices : Methods and applications to spectroscopic data. Appl. Stat.-J. Roy. St. C, 44:101--115.

1995: P. H. Taylor, \textbf{P. Jonathan}, and L. A. Harland. Time-domain simulation of jack-up dynamics using the extremes of a random process. Trans. ASME : J. Vibrat. Accoust. 119:624.

1995: R. W. F. Welling, \textbf{P. Jonathan}, R. H. F. Reijnen, and A. J. Samuel. Quantifying the factors influencing gravel placement and productivity of an internally gravel-packed completion based on field data analysis. Society of Petroleum Engineers, 30113.

1994: M. Stone and \textbf{P. Jonathan}. Statistical thinking and technique for QSAR and related studies. part II : Specific methods. J. Chemometr. 8:1--20.

1993: M. Stone and \textbf{P. Jonathan}. Statistical thinking and technique for QSAR and related studies. part I : General theory. J. Chemometr. 7:455-475.

1988: M. Hamdan, \textbf{P. Jonathan}, R. G. Kingston, and J. H. Beynon. Single-electron capture by $Cl^{++}$ (4S, 2D, 2P) from rare-gas targets. Int. J. Mass Spec. Ion Proc. 83:331.

1988: Z. Herman, \textbf{P. Jonathan}, A. G. Brenton, and J.H.Beynon. Non-dissociative single electron capture by $NO{++}$ from noble gases. Chem. Phys. 123:377.

1988: \textbf{P. Jonathan}, A. G. Brenton, and J. H. Beynon. The effect of charge-exchange collision gas on "resolved" E/2 mass spectra. Organic Mass Spectrometry, 23:115.

1988: \textbf{P. Jonathan}, M. Hamdan, A. G. Brenton, and G. D. Willett. Translational spectroscopy of the triatomic dications $CO_2^{++}$, $OCS^{++}$ and $CS_2^{++}$. Chem. Phys. 119:159.

1987: P. G. Fournier, A. G. Brenton, \textbf{P. Jonathan}, and J. H. Beynon. Dissociation of $H_2^+$ prepared in different rovibronic states by keV collisions with rare gases. Int. J. Mass Spec. Ion Proc. 79:81.

1987: \textbf{P. Jonathan}, A. G. Brenton, J. H. Beynon, and R. K. Boyd. Diatomic dications of noble gas chlorides. Int. J. Mass Spec. Ion Proc. 72:319.

1987: \textbf{P. Jonathan}, Z. Herman, M. Hamdan, and A. G. Brenton. Translational energy spectroscopy of the $NO^{++}$ dication. Chem. Phys. Lett. 141:511.

1987: \textbf{P. Jonathan}, A. R. Lee, A. G. Brenton, and J. H. Beynon. Capture dissociation of $H_2^+$ in rare gases and small hydrocarbons. Int. J. Mass Spec. Ion Proc. 79:101.

1987: Z. Herman, \textbf{P. Jonathan}, A. G. Brenton, and J. H. Beynon. Non-dissociative single electron capture by $CO^{++}$ from rare gases. Chem. Phys. Lett. 141:433.

1987: E. Y. Kamber, \textbf{P. Jonathan}, A. G. Brenton, and J. H. Beynon. Single-electron capture by $Ar^{++}$ from atomic and molecular targets. J. Phys. B: Atom. Mol. Physics, 20:4129.

1987: A. R. Lee, \textbf{P. Jonathan}, A. G. Brenton, and J. H. Beynon. Dissociative electron capture of $H_2^+$ into H fragments. Int. J. Mass Spec. Ion Proc. 75:329.

1987: A. R. Lee, \textbf{P. Jonathan}, A. G. Brenton, and J. H. Beynon. Translational energy loss of $H^+$ fragments from capture-dissociation of $H_2^+$ in collision with rare gas atoms. Phys. Lett. A, 122:346.

1986: \textbf{P. Jonathan}, A. G. Brenton, J. H. Beynon, and R. K. Boyd. Characterisation of cluster ions formed from inert gas atoms plus small molecules and radicals. Int. J. Mass Spec. Ion Proc. 71:257.

1986: \textbf{P. Jonathan}, R. K. Boyd, A. G. Brenton, and J. H. Beynon. Diatomic dications containing one inert gas atom. Chem. Phys. 110:239.

1986: \textbf{P. Jonathan}, R. K. Boyd, A. G. Brenton, and J. H. Beynon. Interference peaks in translational energy loss spectra. An example in the spectrum of $NO^{++}$ . Int. J. Mass Spec. Ion Proc. 68:91.

\setlength{\parskip}{0em}

%\pagebreak
\section{Articles in preparation}

\setlength{\parskip}{0.5em}

J. P. Grainger, A. Sykulski a \textbf{P. Jonathan}. Amcangyfrif sbectrwm amleddol tonnau disgyrchiol arwyneb y m\^or o amgylch arfordir Cymru.

C. J. R. Murphy-Barltrop, E. Mackay and \textbf{P. Jonathan}. A novel inference framework for the SPAR model.

Y. Wu, J. A. Tawn and \textbf{P. Jonathan}. Characterising large-scale spatial extremal dependence of storm environments in the North Sea.

\setlength{\parskip}{0em}

\section{Yn y Gymraeg}
Modelydd ystadegol yng nghwmni Shell ac Adran Fathemateg ac Ystagedd Prifysgol Caerhirfryn yw Philip Jonathan. Mae'n astudio sistemau amgylcheddol yn cynnwys eithafon stormydd morol a gwasgariad nwyon atmosfferig gan ddefnyddio dulliau gwerthoedd eithafol a gofodol-amserol Bayesaidd. Am rhagor o wybodaeth, ewch i \texttt{www.lancs.ac.uk/$\sim$jonathan}.

\clearpage
\end{document}
%% end of file `template.tex'.
